\documentclass[../main.tex]{subfile}
\begin{document}
    The goal of this project is to provide a global wind system for your games inspiered by the laws of electrostatics. After placing 
    some wind source and well you will obtain a three dimensional force field that can be used to move the object and the trees around.
    The wind direction can then make some spirals etc and is complexer than a global wind direction for the map.
    For the fine tuning you can add some boosters at certain locations and give them a direction to boost at. The trees and the particle
    system can follow the wind movement resulting in a great world dynamic.\\

    The system can evolve with the time if you want or just be static an calculated at the beggining. The complexity of the
    calculation grows linearly with the number of source/well/boosters.
    A 2D texture carries the force field to the shader to allow the verticies of your trees, fabric etc to move in the correct direction.
    Moreover a phase factor can be taken into account to allow different response to the wind field depending on the material.
    For exemple a bigger tree will react with a small delay compared to a fir tree.
\end{document}